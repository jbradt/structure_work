\documentclass{article}

\usepackage{amsmath}
\usepackage{mathtools}
\usepackage{amsfonts}
\usepackage{url}
\usepackage{xspace}
\usepackage{siunitx}
\usepackage{cancel}
\usepackage[usenames,dvipsnames]{xcolor}
\usepackage{tikz}
\usepackage{float}

\usepackage{vletters}
%\usepackage{simplewick}
\usepackage{wick}

\usepackage{enumitem}
\setlist[enumerate]{label=(\alph*)}

% Formatting options 
\frenchspacing
% \setlength{\parindent}{0 ex}
% \setlength{\parskip}{3 ex plus 2 ex minus 1 ex}

% Defined macros

\DeclareMathOperator{\csch}{csch}
\DeclareMathOperator{\sech}{sech}
\DeclareMathOperator{\perm}{\mathit{\hat{P}}}

\newcommand{\degree}[0]{\ensuremath{^\circ}\xspace}
\renewcommand{\implies}{\Rightarrow}
\newcommand{\eval}[1]{\ensuremath{\left<#1\right>}}
\newcommand{\ket}[1]{\ensuremath{\left| #1 \right>}}
\newcommand{\bra}[1]{\ensuremath{\left< #1 \right|}}
\newcommand{\mel}[3]{\ensuremath{\left<#1 \right| #2 \left| #3 \right>}}
\newcommand{\proj}[2]{\ensuremath{\left<#1 \middle| #2 \right>}}

\newcommand{\pmat}[1]{\ensuremath{\begin{pmatrix}#1\end{pmatrix}}}

\newcommand{\su}[0]{\ensuremath{\uparrow}}
\newcommand{\sd}[0]{\ensuremath{\downarrow}}

\newcommand{\pder}[2]{\ensuremath{\frac{\partial #1}{\partial #2}}}
\newcommand{\ppder}[2]{\ensuremath{\frac{\partial^2 #1}{\partial #2^2}}}
\newcommand{\ppmder}[3]{\ensuremath{\frac{\partial^2 #1}{\partial #2 \partial #3}}}

\newcommand{\pderc}[3]{\ensuremath{\left( \frac{\partial #1}{\partial #2} \right)_{\!\!#3}}}
\newcommand{\ppmderc}[4]{\ensuremath{\left( \frac{\partial^2 #1}{\partial #2 \partial #3} \right)_{\!\!#4}}}

\newcommand{\phias}[0]{\ensuremath{\Phi^\text{AS}}}
\newcommand{\phisub}[2]{\ensuremath{\phi_{#1}\!\left(#2\right)}}
\newcommand{\psisub}[2]{\ensuremath{\psi_{#1}\!\left(#2\right)}}
\newcommand{\phisubs}[2]{\ensuremath{\phi^*_{#1}\!\left(#2\right)}}
\newcommand{\psisubs}[2]{\ensuremath{\psi^*_{#1}\!\left(#2\right)}}

\newcommand{\ah}[1]{\ensuremath{a_{#1}}}
\newcommand{\ad}[1]{\ensuremath{a^{\dagger}_{#1}}}

% Titles and headers

\title{Phy 981 Assignment 3}
\author{Josh Bradt}
\date{February 11, 2015}

\makeatletter
\let\thetitle\@title
\let\theauthor\@author
\makeatother

\usepackage{fancyhdr}
\pagestyle{fancy}
\chead{\footnotesize \MakeUppercase{\thetitle}} \rhead{\footnotesize\thepage}
\cfoot{}
\renewcommand{\headrulewidth}{0pt}

\begin{document}

\maketitle

\section*{Exercise 5}

	\begin{enumerate}
		\item Using bra-ket notation, the transformation can be written as follows:
		\begin{equation}
			\ket{\psi_a} = \sum_{\lambda} C_{a \lambda} \ket{\phi_\lambda}  \label{eq:0dtrans}
		\end{equation}
		This makes the orthogonality integral easy to compute.
		\begin{align*}
			\proj{\psi_b}{\psi_a} &= \sum_{\lambda \nu} C_{a \lambda} C^*_{\nu b} \proj{\phi_\nu}{\phi_\lambda} \\
								  &= \sum_{\lambda \nu} C_{a \lambda} C^*_{\nu b} \delta_{\lambda\nu} \\
								  &= \sum_{\lambda} C_{a \lambda} C^*_{\lambda b} \\
								  &= (\mathbf{C}\mathbf{C}^*)_{ab} = \delta_{ab}
		\end{align*}
		The last step follows from the unitarity of the matrix $\mathbf{C}$. This proves that the new basis is orthonormal.

		Equation (\ref{eq:0dtrans}) shows the transformation for one basis wavefunction. This can be extended intuitively to a one-dimensional vector of wavefunctions,
		\begin{equation}
			\pmat{\psi_a \\ \psi_b \\ \vdots \\ \psi_A} = 
				\begin{pmatrix}
					C_{a \lambda} & C_{a \mu} & \dots & C_{a A} \\
					C_{b \lambda} & C_{b \mu} & \dots & C_{b A} \\
					\vdots        & \vdots    & \ddots& \vdots  \\
					C_{A \lambda} & C_{A \mu} & \dots & C_{A A} \\
				\end{pmatrix}
				\pmat{\phi_\lambda \\ \phi_\mu \\ \vdots \\ \phi_A}.  \label{eq:1dtrans}
		\end{equation}
		Importantly, note that the wavefunctions $\{\phi_a, \phi_b, \dots\}$ above really mean $\{\phi_a (x_1), \phi_b (x_1), \dots\}$. This implies that if we vary the argument $x_i$ to the wavefunctions, we can extend (\ref{eq:1dtrans}) to two dimensions:
		\begin{multline}
			\begin{pmatrix}
				\psi_a (x_1) & \psi_a (x_2) & \dots & \psi_a(x_A) \\
				\psi_b (x_1) & \psi_b (x_2) & \dots & \psi_b(x_A) \\
				\vdots       & \vdots       & \ddots& \vdots      \\
				\psi_A (x_1) & \psi_A (x_2) & \dots & \psi_A(x_A) \\
			\end{pmatrix} = \\
			\begin{pmatrix}
				C_{a \lambda} & C_{a \mu} & \dots & C_{a A} \\
				C_{b \lambda} & C_{b \mu} & \dots & C_{b A} \\
				\vdots        & \vdots    & \ddots& \vdots  \\
				C_{A \lambda} & C_{A \mu} & \dots & C_{A A} \\
			\end{pmatrix}
			\begin{pmatrix}
			\phi_\lambda (x_1) & \phi_\lambda (x_2) & \dots & \phi_\lambda(x_A) \\
			\phi_\mu (x_1) & \phi_\mu (x_2) & \dots & \phi_\mu(x_A) \\
			\vdots       & \vdots       & \ddots& \vdots      \\
			\phi_A (x_1) & \phi_A (x_2) & \dots & \phi_A(x_A) \\
		\end{pmatrix}  \label{eq:2dtrans}
		\end{multline}
		The rules of matrix multiplication ensure that we will get the same result for each of the column vectors as we would in (\ref{eq:1dtrans}). 

		Now, for brevity, let (\ref{eq:2dtrans}) be written as
		\begin{equation}
			\mathbf{\Psi} = \mathbf{C} \mathbf{\Phi}.
		\end{equation}
		The matrices $\mathbf{\Psi}$ and $\mathbf{\Phi}$ are exactly the same as the matrices we would use to find the Slater determinants in the new and old bases, respectively. Thus,
		\begin{equation*}
			\ket{\Psi} = \det (\mathbf{\Psi}) = \det(\mathbf{C} \mathbf{\Phi}) = \det(\mathbf{C}) \det(\mathbf{\Phi}) = \det(\mathbf{C}) \ket{\Phi}
		\end{equation*}
		The second-to-last step above is valid because $\mathbf{C}$ and $\mathbf{\Phi}$ are square matrices. Therefore, the Slater determinant in the new basis is equal to the determinant in the old basis times the determinant of the transformation matrix.

		Finally, since $\mathbf{C}$ is a unitary matrix, $|\det(\mathbf{C})| = 1$ by definition. Therefore, the two Slater determinants are equal up to a phase.

		\item The normalization integral is:
		\begin{align*}
			\proj{\Phi_0}{\Phi_0} &= \prod_{i=1}^n \prod_{j=1}^n \mel{0}{a_{\alpha_i} a^\dagger_{\alpha_j}}{0} \\
								  &= \prod_{i=1}^n \prod_{j=1}^n \left[ \cancelto{0}{\mel{0}{\left\{ a_{\alpha_i} a^\dagger_{\alpha_j} \right\}}{0}} + \mel{0}{\left\{ \wick{1}{<1a_{\alpha_i} >1a^\dagger_{\alpha_j}} \right\}}{0} \right] \\
								  &= \prod_{i=1}^n \prod_{j=1}^n \delta_{ij} = \delta_{ij}
		\end{align*}
	\end{enumerate}

\section*{Exercise 6}

	For the one-body matrix element, use Wick's theorem:
	\begin{equation*}
		\mel{\alpha_1\alpha_2}{\hat F}{\alpha_1 \alpha_2} = \sum_{\alpha\beta} \mel{\alpha}{f}{\beta} \mel{0}{\ah{\alpha_2}\ah{\alpha_1}\ad{\alpha}\ah{\beta}\ad{\alpha_1}\ad{\alpha_2}}{0}.
	\end{equation*}
	To keep things simple, I'll organize the contractions in a table:
	\begin{table}[H]
		\centering
		\begin{tabular}{c | c}
			$\wick{211}{<1a_{\alpha_2} <2a_{\alpha_1} >2a^\dagger_\alpha <3a_\beta >1a^\dagger_{\alpha_1} >3a^\dagger_{\alpha_2}}$ & $-\delta_{\alpha_1\alpha_2} \delta_{\alpha\alpha_1} \delta_{\beta\alpha_2}$ \\
			$\wick{121}{<1a_{\alpha_2} <2a_{\alpha_1} >1a^\dagger_\alpha <3a_\beta >2a^\dagger_{\alpha_1} >3a^\dagger_{\alpha_2}}$ & $\delta_{\alpha\alpha_2} \delta_{\beta\alpha_2}$ \\
			$\wick{211}{<1a_{\alpha_2} <2a_{\alpha_1} >2a^\dagger_\alpha <3a_\beta >3a^\dagger_{\alpha_1} >1a^\dagger_{\alpha_2}}$ & $\delta_{\alpha\alpha_1} \delta_{\beta\alpha_1}$ \\
			$\wick{121}{<1a_{\alpha_2} <2a_{\alpha_1} >1a^\dagger_\alpha <3a_\beta >3a^\dagger_{\alpha_1} >2a^\dagger_{\alpha_2}}$ & $-\delta_{\alpha_1\alpha_2} \delta_{\alpha\alpha_2} \delta_{\beta\alpha_1}$ \\
		\end{tabular}
	\end{table}
	\noindent Since $\alpha_1 \neq \alpha_2$, only the second and third terms are nonzero. Thus,
	\begin{gather}
		\boxed{\mel{\alpha_1\alpha_2}{\hat F}{\alpha_1 \alpha_2} = \mel{\alpha_1}{\hat f}{\alpha_2} + \mel{\alpha_2}{\hat f}{\alpha_2}}
	\end{gather}
	just as in the previous exercises.

	For the two-body operator,
	\begin{equation*}
		\mel{\alpha_1\alpha_2}{\hat G}{\alpha_1\alpha_2} = \frac{1}{4} \sum_{\alpha\beta\gamma\delta} \mel{\alpha\beta}{g}{\gamma\delta} 
			\mel{0}{a_{\alpha_2} a_{\alpha_1} a^\dagger_{\alpha} a^\dagger_\beta a_\delta a_\gamma a^\dagger_{\alpha_1} a^\dagger_{\alpha_2}}{0}.
	\end{equation*}
	The nonzero contractions are:
	\begin{table}[H]
		\centering
		\begin{tabular}{c | c}
			$\wick{1212}{<1a_{\alpha_2} <2a_{\alpha_1} >1a^\dagger_\alpha >2a^\dagger_\beta <3a_\delta <4a_\gamma >3a^\dagger_{\alpha_1} >4a^\dagger_{\alpha_2}}$ 
				& $\delta_{\alpha\alpha_2} \delta_{\beta\alpha_1} \delta_{\gamma\alpha_2} \delta_{\delta \alpha_1}$ \\
			$\wick{1221}{<1a_{\alpha_2} <2a_{\alpha_1} >1a^\dagger_\alpha >2a^\dagger_\beta <3a_\delta <4a_\gamma >4a^\dagger_{\alpha_1} >3a^\dagger_{\alpha_2}}$ 
				& $-\delta_{\alpha\alpha_2} \delta_{\beta\alpha_1} \delta_{\gamma\alpha_1} \delta_{\delta \alpha_2}$ \\
			$\wick{2112}{<1a_{\alpha_2} <2a_{\alpha_1} >2a^\dagger_\alpha >1a^\dagger_\beta <3a_\delta <4a_\gamma >3a^\dagger_{\alpha_1} >4a^\dagger_{\alpha_2}}$
				& $-\delta_{\alpha\alpha_1} \delta_{\beta\alpha_2} \delta_{\gamma\alpha_2} \delta_{\delta \alpha_1}$ \\
			$\wick{2121}{<1a_{\alpha_2} <2a_{\alpha_1} >2a^\dagger_\alpha >1a^\dagger_\beta <3a_\delta <4a_\gamma >4a^\dagger_{\alpha_1} >3a^\dagger_{\alpha_2}}$
				& $\delta_{\alpha\alpha_1} \delta_{\beta\alpha_2} \delta_{\gamma\alpha_1} \delta_{\delta \alpha_2}$ \\
		\end{tabular}
	\end{table}
	\noindent This means that
	\begin{multline*}
		\mel{\alpha_1\alpha_2}{\hat G}{\alpha_1\alpha_2} = \frac{1}{4} \left\{
			 \mel{\alpha_2\alpha_1}{g}{\alpha_2\alpha_1}
			-\mel{\alpha_2\alpha_1}{g}{\alpha_1\alpha_2} \right. \\ \left.
	 	    -\mel{\alpha_1\alpha_2}{g}{\alpha_2\alpha_1}
			+\mel{\alpha_1\alpha_2}{g}{\alpha_1\alpha_2}
			\right\}
	\end{multline*}
	Using the symmetries of the antisymmetrized matrix elements, this reduces to
	\begin{equation}
		\boxed{\mel{\alpha_1\alpha_2}{\hat G}{\alpha_1\alpha_2} = \mel{\alpha_1\alpha_2}{g}{\alpha_1\alpha_2}_\text{AS}}
	\end{equation}
	which is the same result as we found last week.

\section*{Exercise 7}
	This form of the one-body part of the Hamiltonian can be found using Wick's theorem.
	\begin{align*}
		\hat H_0 &= \sum_{pq} \mel{p}{h_0}{q} a^\dagger_p a_q \\
				 &= \sum_{pq} \mel{p}{h_0}{q} \left[ \left\{ a^\dagger_p a_q \right\} + \left\{ \wick{1}{<1a^\dagger_p >1a_q} \right\} \right]  \\
				 &= \sum_{pq} \mel{p}{h_0}{q} \left[ \left\{ a^\dagger_p a_q \right\} + \delta_{pq \in i} \right]  \\
				 &= \sum_{pq} \mel{p}{h_0}{q} \left\{ a^\dagger_p a_q \right\} + \sum_{i} \mel{i}{h_0}{i} 
	\end{align*}
	Here, $p$ and $q$ represent general one-particle states, and $i$ represents a hole state inside the closed shell. The symbol $\delta_{pq \in i}$ is defined as follows:
	\begin{equation}
		\delta_{pq \in i} = \begin{cases*}
			1 & if $p = q$ and $p, q \leq F$ \\
			0 & otherwise \\
			\end{cases*}
		\label{delta}
	\end{equation}
	In this situation, the reference vacuum is a some closed-core shell that is relevant to the problem at hand.

\section*{Exercise 8}
	We can use the same method as above for the two-body operator, but the calculation is a bit more complicated.
	\begin{equation*}
		\hat H_I = \frac{1}{4} \sum_{pqrs} \mel{pq}{v}{rs} a^\dagger_p a^\dagger_q a_s a_r
	\end{equation*}
	To use Wick's theorem, we'll need all of the possible contractions. I'll list them in the table below along with their associated terms (neglecting the $1/4$).
	\begin{table}[H]
		\centering
		\begin{tabular}{c | c | l}
			$\wick{1}{a^\dagger_p <1a^\dagger_q >1a_s a_r}$ & $\delta_{qs \in i}$ & $+\mel{pi}{v}{ri} a^\dagger_p a_r$ \\
			$\wick{1}{<1a^\dagger_p a^\dagger_q a_s >1a_r}$ & $\delta_{pr \in i}$ & $+\mel{iq}{v}{is} a^\dagger_q a_s$\\
			$\wick{1}{a^\dagger_p <1a^\dagger_q a_s >1a_r}$ & $-\delta_{qr \in i}$ & $-\mel{pi}{v}{is} a^\dagger_p a_s$\\
			$\wick{1}{<1a^\dagger_p a^\dagger_q >1a_s a_r}$ & $-\delta_{ps \in i}$ & $-\mel{iq}{v}{ri} a^\dagger_q a_r$\\
			$\wick{21}{<1a^\dagger_p <2a^\dagger_q >2a_s >1a_r}$ & $\delta_{pr \in i} \delta_{qs \in j}$ & $+\mel{ij}{v}{ij}$ \\
			$\wick{12}{<1a^\dagger_p <2a^\dagger_q >1a_s >2a_r}$ & $-\delta_{ps \in i} \delta_{qr \in j}$ & $-\mel{ij}{v}{ji}$ \\
		\end{tabular}
	\end{table}
	\noindent The first four terms in the table can be combined by using the symmetry of the antisymmetric matrix element and the fact that the labels $p,q,r,s$ are arbitrary. The last two terms can also be combined using symmetry. These combine with the uncontracted term to give the final answer:
	\begin{equation*}
		\boxed{\hat H_I = \frac{1}{4} \sum_{pqrs} \mel{pq}{v}{rs} \left\{ a^\dagger_p a^\dagger_q a_s a_r \right\} + \sum_{pqi} \mel{pi}{v}{qi} \left\{ a^\dagger_p a_q \right\} + \frac{1}{2} \sum_{ij} \mel{ij}{v}{ij}}
	\end{equation*}
	Here, $p,q,r,s$ are general one-particle states, $i,j$ represent hole states below the Fermi level, and $\delta_{pq \in i}$ is defined as above in (\ref{delta}).

\end{document}