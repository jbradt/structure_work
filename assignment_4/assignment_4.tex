\documentclass{article}

\usepackage{amsmath}
\usepackage{mathtools}
\usepackage{amsfonts}
\usepackage{url}
\usepackage{xspace}
\usepackage{siunitx}
\usepackage{cancel}
\usepackage[usenames,dvipsnames]{xcolor}
\usepackage{tikz}
\usepackage{float}

\usepackage{vletters}
%\usepackage{simplewick}
\usepackage{wick}

\usepackage{enumitem}
\setlist[enumerate]{label=(\alph*)}

% Formatting options 
\frenchspacing
% \setlength{\parindent}{0 ex}
% \setlength{\parskip}{3 ex plus 2 ex minus 1 ex}

% Defined macros

\DeclareMathOperator{\csch}{csch}
\DeclareMathOperator{\sech}{sech}
\DeclareMathOperator{\perm}{\mathit{\hat{P}}}

\newcommand{\degree}[0]{\ensuremath{^\circ}\xspace}
\renewcommand{\implies}{\Rightarrow}
\newcommand{\eval}[1]{\ensuremath{\left<#1\right>}}
\newcommand{\ket}[1]{\ensuremath{\left| #1 \right>}}
\newcommand{\bra}[1]{\ensuremath{\left< #1 \right|}}
\newcommand{\mel}[3]{\ensuremath{\left<#1 \right|\! #2 \!\left| #3 \right>}}
\newcommand{\proj}[2]{\ensuremath{\left<#1 \middle| #2 \right>}}

\newcommand{\pmat}[1]{\ensuremath{\begin{pmatrix}#1\end{pmatrix}}}

\newcommand{\su}[0]{\ensuremath{\uparrow}}
\newcommand{\sd}[0]{\ensuremath{\downarrow}}

\newcommand{\pder}[2]{\ensuremath{\frac{\partial #1}{\partial #2}}}
\newcommand{\ppder}[2]{\ensuremath{\frac{\partial^2 #1}{\partial #2^2}}}
\newcommand{\ppmder}[3]{\ensuremath{\frac{\partial^2 #1}{\partial #2 \partial #3}}}

\newcommand{\pderc}[3]{\ensuremath{\left( \frac{\partial #1}{\partial #2} \right)_{\!\!#3}}}
\newcommand{\ppmderc}[4]{\ensuremath{\left( \frac{\partial^2 #1}{\partial #2 \partial #3} \right)_{\!\!#4}}}

\newcommand{\phias}[0]{\ensuremath{\Phi^\text{AS}}}
\newcommand{\phisub}[2]{\ensuremath{\phi_{#1}\!\left(#2\right)}}
\newcommand{\psisub}[2]{\ensuremath{\psi_{#1}\!\left(#2\right)}}
\newcommand{\phisubs}[2]{\ensuremath{\phi^*_{#1}\!\left(#2\right)}}
\newcommand{\psisubs}[2]{\ensuremath{\psi^*_{#1}\!\left(#2\right)}}

\newcommand{\ah}[1]{\ensuremath{a_{#1}}}
\newcommand{\ad}[1]{\ensuremath{a^{\dagger}_{#1}}}

% Titles and headers

\title{Phy 981 Assignment 4}
\author{Josh Bradt}
\date{February 18, 2015}

\makeatletter
\let\thetitle\@title
\let\theauthor\@author
\makeatother

\usepackage{fancyhdr}
\pagestyle{fancy}
\chead{\footnotesize \MakeUppercase{\thetitle}} \rhead{\footnotesize\thepage}
\cfoot{}
\renewcommand{\headrulewidth}{0pt}

\begin{document}

\maketitle

\section*{Exercise 9}

	\begin{enumerate}
		\item The two states $\ket{\Phi_i^a}$ and $\ket{\Phi_{ij}^{ab}}$ can be written in terms of the new reference vacuum $\ket{\Phi_0}$ as follows:
		\begin{gather*}
			\boxed{\ket{\Phi_i^a} = a^\dagger_a a_i \ket{\Phi_0}} \\
			\boxed{\ket{\Phi_{ij}^{ab}} = a^\dagger_a a^\dagger_b a_j a_i \ket{\Phi_0}}
		\end{gather*}

		\item The one-body expectation value is
		\begin{gather*}
			\mel{\Phi_0}{\hat F_N}{\Phi_0} = \sum_{pq} \mel{p}{f}{q} \mel{\Phi_0}{\{a^\dagger_p a_q\}}{\Phi_0} = \boxed{0}
		\end{gather*}
		since the expectation value of a normal-ordered string of operators is zero by definition. 

		Similarly, for the two-body operator,
		\begin{gather*}
			\mel{\Phi_0}{\hat G_N}{\Phi_0} = \frac{1}{4} \sum_{pqrs} \mel{pq}{g}{rs}_\text{AS} \mel{\Phi_0}{\{a^\dagger_p a^\dagger_q a_s a_r\}}{\Phi_0} = \boxed{0.}
		\end{gather*}

		\item For the one-body operator and the one-particle--one-hole excitation, 
		\begin{align*}
			\mel{\Phi_0}{\hat F_N}{\Phi_i^a} &= \sum_{pq} \mel{p}{f}{q} \mel{\Phi_0}{\{a^\dagger_p a_q\} a^\dagger_a a_i}{\Phi_0} \\
				 &= \sum_{pq} \mel{p}{f}{q} \mel{\Phi_0}{\wick{21}{\{<1a^\dagger_p <2a_q\} >2a^\dagger_a >1a_i}}{\Phi_0} \\
				 &= \sum_{pq} \mel{p}{f}{q} \delta_{ip} \delta_{aq} = \boxed{\mel{i}{f}{a}.}
		\end{align*}
		
		For the two-body operator,
		\begin{equation*}
			\mel{\Phi_0}{\hat G_N}{\Phi_{ij}^{ab}} = \frac{1}{4} \sum_{pqrs} \mel{pq}{g}{rs}_\text{AS} \mel{\Phi_0}{\{a^\dagger_p a^\dagger_q a_s a_r\} a^\dagger_a a_i}{\Phi_0} = \boxed{0}
		\end{equation*}
		since there is no way to make a fully contracted term without contracting two operators that are both inside the normal-ordered string.
		
		\item For the one-body operator and the two-particle--two-hole excitation,
		\begin{align*}
			\mel{\Phi_0}{\hat F_N}{\Phi_i^a} &= \sum_{pq} \mel{p}{f}{q} \mel{\Phi_0}{\{a^\dagger_p a_q\} a^\dagger_a a^\dagger_b a_j a_i}{\Phi_0} = \boxed{0}
		\end{align*}
		for the same reason as above.

		For the two-body operator,
		\begin{equation*}
			\mel{\Phi_0}{\hat G_N}{\Phi_{ij}^{ab}} = \frac{1}{4} \sum_{pqrs} \mel{pq}{g}{rs}_\text{AS} \mel{\Phi_0}{\{a^\dagger_p a^\dagger_q a_s a_r\} a^\dagger_a a^\dagger_b a_j a_i}{\Phi_0}
		\end{equation*}
		Here, there are four non-zero contractions:
		\begin{table}[H]
			\centering
			\begin{tabular}{c | c | c}
				$\wick{4321}{\{<1a^\dagger_p <2a^\dagger_q <3a_s <4a_r\} >4a^\dagger_a >3a^\dagger_b >2a_j >1a_i}$ 
					& $\delta_{pi}\delta_{qj}\delta_{sb}\delta_{ra}$ & $\frac{1}{4}\mel{ij}{g}{ab}_\text{AS}$ \\
				$\wick{4321}{\{<1a^\dagger_p <2a^\dagger_q <3a_s <4a_r\} >4a^\dagger_a >3a^\dagger_b >1a_j >2a_i}$ 
					& $-\delta_{pj}\delta_{qi}\delta_{sb}\delta_{ra}$ & $-\frac{1}{4}\mel{ji}{g}{ab}_\text{AS}$ \\
				$\wick{4321}{\{<1a^\dagger_p <2a^\dagger_q <3a_s <4a_r\} >3a^\dagger_a >4a^\dagger_b >2a_j >1a_i}$ 
					& $-\delta_{pi}\delta_{qj}\delta_{sa}\delta_{rb}$ & $-\frac{1}{4}\mel{ij}{g}{ba}_\text{AS}$ \\
				$\wick{4321}{\{<1a^\dagger_p <2a^\dagger_q <3a_s <4a_r\} >3a^\dagger_a >4a^\dagger_b >1a_j >2a_i}$ 
					& $\delta_{pj}\delta_{qi}\delta_{sa}\delta_{rb}$ & $\frac{1}{4}\mel{ji}{g}{ba}_\text{AS}$ \\
			\end{tabular}
		\end{table}
		\noindent Due to the antisymmetry, these four matrix elements are equal. Thus,
		\begin{equation*}
			\mel{\Phi_0}{\hat G_N}{\Phi_{ij}^{ab}} = \boxed{\mel{ij}{g}{ab}_\text{AS}}
		\end{equation*}

		By extension of the fact that a one-body operator produces a result of zero when acting on a two-particle--two-hole state, I would expect that a two-body operator would produce zero when acting on $|\Phi_{ijk}^{abc}\rangle$, a three-particle--three-hole excitation.
	\end{enumerate}

\end{document}