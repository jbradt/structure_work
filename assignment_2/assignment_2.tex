\documentclass{article}

\usepackage{amsmath}
\usepackage{mathtools}
\usepackage{amsfonts}
\usepackage{url}
\usepackage{xspace}
\usepackage{siunitx}
\usepackage{cancel}
\usepackage[usenames,dvipsnames]{xcolor}
\usepackage{tikz}

\usepackage{vletters}

\usepackage{enumitem}
\setlist[enumerate]{label=(\alph*)}

% Formatting options 
\frenchspacing
% \setlength{\parindent}{0 ex}
% \setlength{\parskip}{3 ex plus 2 ex minus 1 ex}

% Defined macros

\DeclareMathOperator{\csch}{csch}
\DeclareMathOperator{\sech}{sech}
\DeclareMathOperator{\perm}{\mathit{\hat{P}}}

\newcommand{\degree}[0]{\ensuremath{^\circ}\xspace}
\renewcommand{\implies}{\Rightarrow}
\newcommand{\eval}[1]{\ensuremath{\left<#1\right>}}
\newcommand{\ket}[1]{\ensuremath{\left| #1 \right>}}
\newcommand{\bra}[1]{\ensuremath{\left< #1 \right|}}
\newcommand{\mel}[3]{\ensuremath{\left<#1 \middle| #2 \middle| #3 \right>}}

\newcommand{\pmat}[1]{\ensuremath{\begin{pmatrix}#1\end{pmatrix}}}

\newcommand{\su}[0]{\ensuremath{\uparrow}}
\newcommand{\sd}[0]{\ensuremath{\downarrow}}

\newcommand{\pder}[2]{\ensuremath{\frac{\partial #1}{\partial #2}}}
\newcommand{\ppder}[2]{\ensuremath{\frac{\partial^2 #1}{\partial #2^2}}}
\newcommand{\ppmder}[3]{\ensuremath{\frac{\partial^2 #1}{\partial #2 \partial #3}}}

\newcommand{\pderc}[3]{\ensuremath{\left( \frac{\partial #1}{\partial #2} \right)_{\!\!#3}}}
\newcommand{\ppmderc}[4]{\ensuremath{\left( \frac{\partial^2 #1}{\partial #2 \partial #3} \right)_{\!\!#4}}}

\newcommand{\phias}[0]{\ensuremath{\Phi^\text{AS}}}
\newcommand{\phisub}[2]{\ensuremath{\phi_{#1}\!\left(#2\right)}}
\newcommand{\psisub}[2]{\ensuremath{\psi_{#1}\!\left(#2\right)}}
\newcommand{\phisubs}[2]{\ensuremath{\phi^*_{#1}\!\left(#2\right)}}
\newcommand{\psisubs}[2]{\ensuremath{\psi^*_{#1}\!\left(#2\right)}}

% Titles and headers

\title{Phy 981 Assignment 2}
\author{Josh Bradt}
\date{January 28, 2015}

\makeatletter
\let\thetitle\@title
\let\theauthor\@author
\makeatother

\usepackage{fancyhdr}
\pagestyle{fancy}
\chead{\footnotesize \MakeUppercase{\thetitle}} \rhead{\footnotesize\thepage}
\cfoot{}
\renewcommand{\headrulewidth}{0pt}

\begin{document}

\maketitle

\section*{Exercise 3}

	\begin{enumerate}
		\item For $N=3$, $\phias$ is
		\begin{align*}
			\phias_\lambda &= \frac{1}{\sqrt{3!}} \sum_p (-)^p \perm \psisub{\alpha_1}{x_1} \psisub{\alpha_2}{x_2} \psisub{\alpha_3}{x_3} \\
			\phias_\lambda &= \begin{multlined}[t]
				\frac{1}{\sqrt{6}} \left[ 
				\psisub{\alpha_1}{x_1} \psisub{\alpha_2}{x_2} \psisub{\alpha_3}{x_3}
				- \psisub{\alpha_1}{x_2} \psisub{\alpha_2}{x_1} \psisub{\alpha_3}{x_3} \right.\\
				- \psisub{\alpha_1}{x_3} \psisub{\alpha_2}{x_2} \psisub{\alpha_3}{x_1}
				- \psisub{\alpha_1}{x_1} \psisub{\alpha_2}{x_3} \psisub{\alpha_3}{x_2} \\
				\left. + \psisub{\alpha_1}{x_3} \psisub{\alpha_2}{x_1} \psisub{\alpha_3}{x_2}
				+ \psisub{\alpha_1}{x_2} \psisub{\alpha_2}{x_3} \psisub{\alpha_3}{x_1}
				\right]
				\end{multlined}
		\end{align*}

		\item The integral can be written out as
		\begin{equation*}
			\int dx_1 \dots dx_N \frac{1}{N!} \left[ \sum_p (-)^p \perm \prod_{i=1}^{N} \psisubs{\alpha_i}{x_i} \right] \left[ \sum_{p} (-)^{p} \perm \prod_{i=1}^{N} \psisub{\alpha_i}{x_i} \right].
		\end{equation*}
		However, since the basis functions $\psisub{\alpha_i}{x_i}$ are orthogonal, the only terms that will survive the integration are those with corresponding permutations from each sum. In other words, the integral can be written as
		\begin{equation*}
			\frac{1}{N!} \int dx_1 \dots dx_N \sum_p (-)^{2p} \perm \prod_{i=1}^{N} \psisubs{\alpha_i}{x_i}\psisub{\alpha_i}{x_i}.
		\end{equation*}
		Now, using the normality of the basis functions, this can be rewritten as follows:
		\begin{gather*}
			\frac{1}{N!} \sum_p \int dx_1 \dots dx_N \perm \prod_{i=1}^{N} \psisubs{\alpha_i}{x_i}\psisub{\alpha_i}{x_i} \\
			=\frac{1}{N!} \sum_p \prod_{i=1}^{N} \int dx_i\;\psisubs{\alpha_i}{x_i}\psisub{\alpha_i}{x_i} \\
			=\frac{1}{N!} \sum_p (1) = \frac{N!}{N!} = 1
		\end{gather*}

		\item For the one-body operator $\hat F = \hat f(x_1) + \hat f(x_2)$, 
		\begin{gather*}
			\mel{\Phi_{\alpha_1 \alpha_2}^{AS}}{\hat f(x_1) + \hat f(x_2)}{\Phi_{\alpha_1 \alpha_2}^{AS}} \\
			= \begin{multlined}[t]
				\frac{1}{2}\int dx_1\;dx_2\; \left[ \psisubs{\alpha_1}{x_1}\psisubs{\alpha_2}{x_2} - \psisubs{\alpha_2}{x_1}\psisubs{\alpha_1}{x_2} \right] \left( \hat f(x_1) + \hat f(x_2) \right) \\
				\times \left[ \psisub{\alpha_1}{x_1}\psisub{\alpha_2}{x_2} - \psisub{\alpha_2}{x_1}\psisub{\alpha_1}{x_2} \right].
				\end{multlined} 
		\end{gather*}
		Since the functions $\psi_{\alpha_i}$ are orthogonal, the cross terms drop out. Thus, the above becomes
		\begin{gather*}
			\begin{multlined}[t]
				\frac{1}{2} \left\{ \int dx_1\;dx_2\; \left[ \psisubs{\alpha_1}{x_1}\psisubs{\alpha_2}{x_2} \right] \left( \hat f(x_1) + \hat f(x_2) \right) \left[ \psisub{\alpha_1}{x_1}\psisub{\alpha_2}{x_2} \right] \right. \\
				+\left. \int dx_1\;dx_2\; \left[ \psisubs{\alpha_2}{x_1}\psisubs{\alpha_1}{x_2} \right] \left( \hat f(x_1) + \hat f(x_2) \right) \left[ \psisub{\alpha_2}{x_1}\psisub{\alpha_1}{x_2} \right] \right\}.
				\end{multlined} 
			\intertext{However, since $x_1$ and $x_2$ are arbitrary bound variables, we can just switch them in the second term and find}
			\frac{1}{2} \left\{ 2 \int dx_1\;dx_2\; \left[ \psisubs{\alpha_1}{x_1}\psisubs{\alpha_2}{x_2} \right] \left( \hat f(x_1) + \hat f(x_2) \right) \left[ \psisub{\alpha_1}{x_1}\psisub{\alpha_2}{x_2} \right] \right\} \\
			= \boxed{\mel{\alpha_1\alpha_2}{\hat f(x_1)}{\alpha_1\alpha_2} + \mel{\alpha_1\alpha_2}{\hat f(x_2)}{\alpha_1\alpha_2}}
		\end{gather*}

		The calculation is similar for the two-body operator. This time,
		\begin{equation*}
			\hat G = \sum_{i>j}^N \hat g(x_i, x_j) = \hat g(x_1, x_2)
		\end{equation*}
		since $\hat g$ is invariant under an exchange of $x_1$ and $x_2$. Then,
		\begin{gather*}
			\mel{\Phi_{\alpha_1 \alpha_2}^{AS}}{\hat g(x_1, x_2)}{\Phi_{\alpha_1 \alpha_2}^{AS}} \\
			= \begin{multlined}[t]
				\frac{1}{2}\int dx_1\;dx_2\; \left[ \psisubs{\alpha_1}{x_1}\psisubs{\alpha_2}{x_2} - \psisubs{\alpha_2}{x_1}\psisubs{\alpha_1}{x_2} \right] \hat g(x_1, x_2) \\
				\times \left[ \psisub{\alpha_1}{x_1}\psisub{\alpha_2}{x_2} - \psisub{\alpha_2}{x_1}\psisub{\alpha_1}{x_2} \right]
				\end{multlined} \\
			= \begin{aligned}[t]
				\frac{1}{2}&\int dx_1\;dx_2\; \psisubs{\alpha_1}{x_1}\psisubs{\alpha_2}{x_2} \hat g(x_1, x_2) \psisub{\alpha_1}{x_1}\psisub{\alpha_2}{x_2} \\
				&+ \frac{1}{2}\int dx_1\;dx_2\; \psisubs{\alpha_2}{x_1}\psisubs{\alpha_1}{x_2} \hat g(x_1, x_2) \psisub{\alpha_2}{x_1}\psisub{\alpha_1}{x_2} \\
				&- \frac{1}{2}\int dx_1\;dx_2\; \psisubs{\alpha_1}{x_1}\psisubs{\alpha_2}{x_2} \hat g(x_1, x_2) \psisub{\alpha_2}{x_1}\psisub{\alpha_1}{x_2} \\
				&- \frac{1}{2}\int dx_1\;dx_2\; \psisubs{\alpha_2}{x_1}\psisubs{\alpha_1}{x_2} \hat g(x_1, x_2) \psisub{\alpha_1}{x_1}\psisub{\alpha_2}{x_2}.
				\end{aligned}
		\intertext{Using the same trick of swapping the indices as above, this becomes}
			\begin{multlined}[t]
				\int dx_1\;dx_2\; \psisubs{\alpha_1}{x_1}\psisubs{\alpha_2}{x_2} \hat g(x_1, x_2) \psisub{\alpha_1}{x_1}\psisub{\alpha_2}{x_2} \\
				- \int dx_1\;dx_2\; \psisubs{\alpha_1}{x_1}\psisubs{\alpha_2}{x_2} \hat g(x_1, x_2) \psisub{\alpha_2}{x_1}\psisub{\alpha_1}{x_2} 
				\end{multlined} \\
			= \mel{\alpha_1\alpha_2}{\hat g(x_1, x_2)}{\alpha_1\alpha_2} - \mel{\alpha_1\alpha_2}{\hat g(x_1, x_2)}{\alpha_2\alpha_1} \\
			= \boxed{\mel{\alpha_1\alpha_2}{\hat g(x_1, x_2)}{\alpha_1\alpha_2}_\text{AS}}
		\end{gather*}

		The shorthand notation for the Slater determinant, $\Phi_{\alpha_1\alpha_2}^{AS}$, indicates that the determinant is antisymmetric and is taken over two particles with quantum numbers $\alpha_1$ and $\alpha_2$.

		In addition to permutation symmetry, I would expect that the one-body operator will be diagonal in the basis of single-particle states, and the two-body operator will be symmetric to reflect the permutation symmetry.

	\end{enumerate}

\section*{Exercise 4}

	\begin{enumerate}
		\item This is a problem of combinatorics. Since the Pauli principle forbids putting more than one particle into a state with a particular $p$ and spin, the number of possible Slater determinants is
		\begin{equation*}
		 	\pmat{6\\2} = \frac{6!}{2!(6-2)!} = \boxed{15.}
		 \end{equation*} 
		If we limit ourselves to only states with paired particles, as below, then there would be three possible Slater determinants.

		\item Based on the assumption that the one-body part of the Hamiltonian has the form 
		\begin{equation}
			\hat h_0 \psi_{p\sigma} = pd \; \psi_{p\sigma},
		\end{equation}
		the matrix elements it produces should be
		\begin{equation}
			\mel{\Phi_0}{\hat H_0}{\Phi_0} = \sum_i \mel{p_i \sigma_i}{h_0}{p_i \sigma_i}.
		\end{equation}
		If we assume that both particles must be in the same state $p$, then this can only take values of $2d$ or $4d$, so
		\begin{equation}
			\hat H_0 = \begin{pmatrix}
				2d & 0  \\
				0  & 4d \\
			\end{pmatrix}
		\end{equation}
		The problem states that the two-body part of the Hamiltonian has matrix elements uniformly equal to $-g$. Therefore, the overall Hamiltonian is
		\begin{equation}
			\hat H = \hat H_0 + \hat H_I = \boxed{\begin{pmatrix}2d-g & -g  \\ -g  & 4d-g \\ \end{pmatrix}}
		\end{equation}

		Using Mathematica, one can show that this matrix has eigenvalues
		\begin{equation}
			\lambda_\pm = 3d - g \pm \sqrt{d^2 + g^2}
		\end{equation}
		and eigenvectors
		\begin{equation}
			v_{\pm} = \left( \frac{d \mp \sqrt{d^2 + g^2}}{g} \quad 1 \right)^T.
		\end{equation}

		The state with two particles in $p=2$ mixes with the state with two particles in $p=1$ through the off-diagonal elements of the Hamiltonian matrix. Thus, 
		\begin{equation}
			\boxed{\mel{22}{H}{11} = -g}
		\end{equation}

		The eigenvectors of the Hamilton can be interpreted as a linear combination of the Slater determinants for two particles in each of the states $p=1$ and $p=2$. Since there is some mixing between these two states (the interaction part of the Hamiltonian), the ``true'' eigenvectors of the system are a linear combination of the basis states of the non-interacting particles.

		\item Based on the math above, we can write, by inspection,
		\begin{equation}
			\hat H = \begin{pmatrix}
				2d-g & -g & -g \\
				-g & 4d-g & -g \\
				-g & -g & 6d-g \\
			\end{pmatrix}
		\end{equation}
		for the case with the $p=3$ state available. There is no closed-form analytic solution to the eigenvalue problem for this matrix, but it can be diagonalized numerically if we choose values for $d$ and $g$. I'll set $d=1$ and look at two values of $g$:
		\begin{description}
			\item[Small g:] The case of small $g$ corresponds to weak interactions between the particles. If we set $g=0.001$, then the matrix is approximately diagonal. The eigenvalues in this case are, to three figures, 2.00, 4.00, and 6.00 (the diagonal of the matrix), and the eigenvectors are 
			\[
				\pmat{1.00\\0.00\\0.00}, \pmat{0.00\\1.00\\0.00},\text{\ and\ } \pmat{0.00\\0.00\\1.00}
			\]
			when taken to the same precision. This is expected since if the particles barely interact, then the single-particle states are a good basis.

			\item[Large g:] If we set $g=10$, then things are much less clean. This produces eigenvalues of $-26.09$, $5.20$, and $2.89$, and eigenvectors
			\[
				\pmat{-0.616\\-0.575\\-0.539}, \pmat{-0.204\\-0.544\\0.814},\text{\ and\ } \pmat{0.761\\-0.611\\-0.218}.
			\]
			Unsurprisingly, a strong interaction term produces a strong mixing of the single-particle states.
		\end{description}
	\end{enumerate}

\end{document}