\documentclass{article}

\usepackage{amsmath}
\usepackage{mathtools}
\usepackage{amsfonts}
\usepackage{url}
\usepackage{xspace}
\usepackage{siunitx}
\usepackage{cancel}
\usepackage[usenames,dvipsnames]{xcolor}
\usepackage{tikz}

\usepackage{vletters}

\usepackage{enumitem}
\setlist[enumerate]{label=(\alph*)}

% Formatting options 
\frenchspacing
% \setlength{\parindent}{0 ex}
% \setlength{\parskip}{3 ex plus 2 ex minus 1 ex}

% Defined macros

\DeclareMathOperator{\csch}{csch}
\DeclareMathOperator{\sech}{sech}
\DeclareMathOperator{\perm}{\mathit{\hat{P}}}

\newcommand{\degree}[0]{\ensuremath{^\circ}\xspace}
\renewcommand{\implies}{\Rightarrow}
\newcommand{\eval}[1]{\ensuremath{\left<#1\right>}}
\newcommand{\ket}[1]{\ensuremath{\left| #1 \right>}}
\newcommand{\bra}[1]{\ensuremath{\left< #1 \right|}}
\newcommand{\mel}[3]{\ensuremath{\left<#1 \middle| #2 \middle| #3 \right>}}

\newcommand{\pder}[2]{\ensuremath{\frac{\partial #1}{\partial #2}}}
\newcommand{\ppder}[2]{\ensuremath{\frac{\partial^2 #1}{\partial #2^2}}}
\newcommand{\ppmder}[3]{\ensuremath{\frac{\partial^2 #1}{\partial #2 \partial #3}}}

\newcommand{\pderc}[3]{\ensuremath{\left( \frac{\partial #1}{\partial #2} \right)_{\!\!#3}}}
\newcommand{\ppmderc}[4]{\ensuremath{\left( \frac{\partial^2 #1}{\partial #2 \partial #3} \right)_{\!\!#4}}}

\newcommand{\phias}[0]{\ensuremath{\Phi^\text{AS}}}
\newcommand{\phisub}[2]{\ensuremath{\phi_{#1}\!\left(#2\right)}}
\newcommand{\psisub}[2]{\ensuremath{\psi_{#1}\!\left(#2\right)}}

% Titles and headers

\title{Phy 981 Assignment 2}
\author{Josh Bradt}
\date{January 28, 2015}

\makeatletter
\let\thetitle\@title
\let\theauthor\@author
\makeatother

\usepackage{fancyhdr}
\pagestyle{fancy}
\chead{\footnotesize \MakeUppercase{\thetitle}} \rhead{\footnotesize\thepage}
\cfoot{}
\renewcommand{\headrulewidth}{0pt}

\begin{document}

\maketitle

\section*{Exercise 3}

	\begin{enumerate}
		\item For $N=3$, $\phias$ is
		\begin{align*}
			\phias_\lambda &= \frac{1}{\sqrt{3!}} \sum_p (-)^p \perm \psisub{\alpha_1}{x_1} \psisub{\alpha_2}{x_2} \psisub{\alpha_3}{x_3} \\
			\phias_\lambda &= \begin{multlined}[t]
				\frac{1}{\sqrt{6}} \left[ 
				\psisub{\alpha_1}{x_1} \psisub{\alpha_2}{x_2} \psisub{\alpha_3}{x_3}
				- \psisub{\alpha_1}{x_2} \psisub{\alpha_2}{x_1} \psisub{\alpha_3}{x_3} \right.\\
				- \psisub{\alpha_1}{x_3} \psisub{\alpha_2}{x_2} \psisub{\alpha_3}{x_1}
				- \psisub{\alpha_1}{x_1} \psisub{\alpha_2}{x_3} \psisub{\alpha_3}{x_2} \\
				\left. + \psisub{\alpha_1}{x_3} \psisub{\alpha_2}{x_1} \psisub{\alpha_3}{x_2}
				+ \psisub{\alpha_1}{x_2} \psisub{\alpha_2}{x_3} \psisub{\alpha_3}{x_1}
				\right]
				\end{multlined}
		\end{align*}

		\item The integral can be written out as
		\begin{equation*}
			\int dx_1 \dots dx_N \frac{1}{N!} \left[ \sum_p (-)^p \perm \prod_{i=1}^{N} \psisub{\alpha_i}{x_i} \right] \left[ \sum_{p} (-)^{p} \perm \prod_{i=1}^{N} \psisub{\alpha_i}{x_i} \right].
		\end{equation*}
		However, since the basis functions $\psisub{\alpha_i}{x_i}$ are orthogonal, the only terms that will survive the integration are those with corresponding permutations from each sum. In other words, the integral can be written as
		\begin{equation*}
			\frac{1}{N!} \int dx_1 \dots dx_N \sum_p (-)^{2p} \perm \prod_{i=1}^{N} \psi_{\alpha_i}^2 (x_i).
		\end{equation*}
		Now, using the normality of the basis functions, this can be rewritten as follows:
		\begin{gather*}
			\frac{1}{N!} \sum_p \int dx_1 \dots dx_N \perm \prod_{i=1}^{N} \psi_{\alpha_i}^2 (x_i) \\
			\frac{1}{N!} \sum_p \prod_{i=1}^{N} \int dx_i \psi_{\alpha_i}^2 (x_i) \\
			\frac{1}{N!} \sum_p (1) = \frac{N!}{N!} = 1
		\end{gather*}
	\end{enumerate}

\end{document}